\renewcommand{\baselinestretch}{1.5}
\fontsize{12pt}{13pt}\selectfont


\chapter{摘~~~~要}
\markboth{中~文~摘~要}{中~文~摘~要}

移动水声传感网络的组网技术是研究水声传感网络的重要方向之一。水声传感网络采用声信号为传播媒介,水声通信时存在高误码率与不可靠性。同时移动节点带来的网络拓扑高动态性也会使人们在开发与设计水声传感网络时面临各种挑战。如何克服这些不利因素的影响,设计合理有效的水声传感网络组网技术与方法,对人类合理开发利用海洋资源、保护国家海洋安全,具有十分重要的意义。

水声传感网络体系结构可以分为三层:网络层、数据链路层、物理层。其中数据链路层主要起着为上层网络协议提供可靠链路作用,它又分为逻辑链路层(LLC)和媒体接入控制层(MAC),本论文以媒体接入控制技术为研究方向。

论文在分析了目前水声传感网络MAC协议的分类和研究现状后,针对移动节点采集多个固定节点数据的单跳水声网络,提出一个应用于移动水声网络的基于MACA和CSMA的负载自适应协议PAMC-M(Playload Adaptive Protocol based on MACA/CSMA for Mobile Underwater Acoustic Network),并进行了性能分析和仿真验证,通过与传统竞争协议UWALOHA和SFAMA对比,得到了一些有意义的研究成果。

论文主要研究成果如下:
\vspace{-10pt}
\begin{enumerate}
	\item 针对少量移动节点接入固定节点网络,单向采集固定节点数据的场景,设计移动节点的接入和离开机制。移动节点按设定时间间隔定时发送BCT广播帧。固定节点根据BCT帧的信息进行相应处理。广播帧的设计使得移动节点可以以较小的代价接入和离开传输网络。
	\item 针对移动水声网络的不可靠性高的特性的特性,提出了数据帧重传机制。数据重传时如果从握手机制重新开始,时间和能耗开销太大。而在协议流程中为单独重传数据帧留出等待时间,可以降低提高传输效率。
	\item 
	针对网络负载较大时控制帧冲突概率较高的问题,提出了自适应负载变化的数据帧发送机制。低负载模式时每次握手传输一次数据帧,高负载模式时传输两次。性能分析和仿真表明,自适应负载变化的发送机制使协议在不同网络负载下表现优秀,有效提高了网络吞吐量,降低了传输延时。
\end{enumerate}
\vspace{-10pt}

\vspace{1em}
\noindent {\fHei 关键词:} \quad 水声传感网络, 媒体接入控制, 竞争型MAC协议,移动节点接入,负载自适应

\clearpage
\endinput
