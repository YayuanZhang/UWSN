\renewcommand{\baselinestretch}{1.5}
\fontsize{12pt}{13pt}\selectfont


\chapter{摘~~~~要}
\markboth{中~文~摘~要}{中~文~摘~要}

移动水声传感网络的组网技术是水声传感网络的重要研究方向之一。水声传感网络以声信号为传播媒介,误码率高、可靠性低。同时移动节点带来的网络拓扑高动态性也使人们在开发与设计水声传感网络时面临各种挑战。克服这些不利因素的影响,设计合理有效的水声传感网络,对人类合理开发利用海洋资源、保护国家海洋安全,具有十分重要的意义。

水声传感网络体系结构可以分为三层:网络层、数据链路层、物理层。其中数据链路层主要起着为上层网络协议提供可靠链路的作用。它又分为逻辑链路层(LLC)和媒体接入控制层(MAC),本论文以媒体接入控制技术为主要研究方向。

论文分析了目前水声传感网络MAC协议的分类和研究现状。针对移动节点采集多个固定节点数据的单跳通信场景,提出了应用于移动水声网络的基于CSMA/CA(Carrier Sense Multiple Access/Collision Avoidance)的负载自适应协议LACC-M(Load Adaptive CSMA/CA for Mobile underwater acoustic network),并进行了性能分析和仿真验证,通过与传统竞争协议UWALOHA和SFAMA对比,得到了一些有意义的成果。

论文主要研究成果如下:
\vspace{-10pt}
\begin{enumerate}
	\item 针对少量移动节点接入固定节点网络、单向采集固定节点数据的场景,设计移动节点的接入和离开机制。移动节点按设定时间间隔定时发送BCT广播帧。固定节点根据BCT帧的信息进行相应处理。广播帧的设计使得移动节点可以以较小的代价接入和离开固定节点网络。
	\item 针对移动水声网络不可靠性高的特性,提出了数据帧重传机制。数据帧重传时如果从握手机制开始重传,时间和能耗开销太大。在协议流程中为单独重传数据帧留出等待时间,则可以提高传输效率。
	\item 
	针对网络负载较大时控制帧冲突概率较高的问题,提出了自适应负载变化的数据帧发送机制。低负载模式时每次握手后传输一次数据帧,高负载模式时传输两次。性能分析和仿真表明,自适应负载变化的发送机制使协议在不同网络负载下表现优秀,有效提高了网络吞吐量,降低了传输延时。
\end{enumerate}
\vspace{-10pt}

\vspace{1em}
\noindent {\fHei 关键词:} \quad 水声传感网络, 媒体接入控制, 竞争型MAC协议,移动节点接入,负载自适应

\clearpage
\endinput
